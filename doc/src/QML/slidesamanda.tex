% ============================================================
% Beamer presentation: ML workflow mapping for the paper
% ============================================================

\documentclass{beamer}
\usetheme{Madrid}
\usepackage{amsmath}
\usepackage{graphicx}

\title[ML Workflow Summary]{Machine Learning Workflow Mapping \\
and Analysis Pipeline for the Paper}
\author{}
\date{}

\begin{document}

% ------------------------------------------------------------
\begin{frame}
\titlepage
\end{frame}

% ------------------------------------------------------------
\begin{frame}{Motivation}
\begin{itemize}
  \item The paper contains high-dimensional structured data.
  \item Scientific goal: identify structure, classify states, and analyze latent geometry.
  \item ML framing: \textbf{classification + representation learning + clustering}.
  \item Topic modeling is not appropriate; the task is not textual or semantic.
\end{itemize}
\end{frame}

% ------------------------------------------------------------
\begin{frame}{Mapping Paper Sections to ML Concepts}
\begin{itemize}
  \item \textbf{Data description} $\rightarrow$ Feature types, dimensionality, labels.
  \item \textbf{Methodology} $\rightarrow$ Supervised learning + latent space analysis.
  \item \textbf{Experiments} $\rightarrow$ Training pipeline, hyperparameters, evaluation.
  \item \textbf{Results} $\rightarrow$ Class separability, PCA/UMAP visualization.
  \item \textbf{Conclusions} $\rightarrow$ Model behavior, robustness, limitations.
\end{itemize}
\end{frame}

% ------------------------------------------------------------
\begin{frame}{Overall ML Workflow}
\begin{enumerate}
  \item Data preprocessing and normalization.
  \item Exploratory data analysis (EDA).
  \item Dimensionality reduction (PCA/UMAP/t-SNE).
  \item Supervised classification models.
  \item Unsupervised clustering models.
  \item Evaluation and cross-validation.
  \item Interpretation of latent geometry.
\end{enumerate}
\end{frame}

% ------------------------------------------------------------
\begin{frame}{Dataset Characteristics}
\begin{itemize}
  \item High-dimensional numerical features.
  \item Potential noise or correlations requiring normalization.
  \item Likely labeled states (multi-class classification).
  \item Latent manifold expected to be low dimensional.
  \item Possible imbalance between categories.
\end{itemize}
\end{frame}

% ------------------------------------------------------------
\begin{frame}{Preprocessing Pipeline}
\begin{itemize}
  \item Standardization: mean-zero, unit variance.
  \item Feature decorrelation: PCA whitening (optional).
  \item Splitting: stratified train/validation/test.
  \item Outlier detection: z-score or Isolation Forest.
  \item Dimensionality reduction before clustering.
\end{itemize}
\end{frame}

% ------------------------------------------------------------
\begin{frame}{Representation Learning}
\begin{itemize}
  \item \textbf{PCA}: linear structure, explained variance.
  \item \textbf{UMAP}: nonlinear manifold discovery, local/global structure.
  \item \textbf{t-SNE}: high-resolution clustering visualization.
  \item \textbf{Autoencoders}: unsupervised compression + feature extraction.
  \item \textbf{Contrastive Learning (SimCLR)}: robust embeddings for classification.
\end{itemize}
\end{frame}

% ------------------------------------------------------------
\begin{frame}{Supervised Learning Algorithms}
Recommended classifiers:
\begin{itemize}
  \item Logistic Regression (baseline, quick interpretability).
  \item Support Vector Machines (RBF kernel for nonlinear boundaries).
  \item Random Forests and XGBoost (feature importance).
  \item Multi-layer Perceptron (deep fully-connected nets).
  \item If relational structure appears: Graph Neural Networks.
\end{itemize}
\end{frame}

% ------------------------------------------------------------
\begin{frame}{Deep Learning Architectures}
\begin{itemize}
  \item Fully-connected feedforward networks with dropout.
  \item Autoencoder encoder + classifier head.
  \item Contrastive learning backbone + linear probe.
  \item Attention-based models if sequence structure exists.
\end{itemize}
\end{frame}

% ------------------------------------------------------------
\begin{frame}{Unsupervised Clustering Methods}
\begin{itemize}
  \item k-means for centroid-based clusters.
  \item Gaussian Mixture Models for soft cluster assignments.
  \item Spectral clustering for data on manifolds.
  \item HDBSCAN for non-spherical, density-based clusters.
  \item Evaluate against labels using: ARI, NMI, homogeneity score.
\end{itemize}
\end{frame}

% ------------------------------------------------------------
\begin{frame}{Evaluation Strategy}
\begin{itemize}
  \item Metrics: accuracy, precision, recall, macro-F1.
  \item Confusion matrix for class-wise error.
  \item k-fold cross-validation for robustness.
  \item Latent space interpretation: PCA/UMAP colored by labels.
  \item Silhouette and ARI scores for cluster validation.
\end{itemize}
\end{frame}

% ------------------------------------------------------------
\begin{frame}{Visualization Tools}
\begin{itemize}
  \item Feature correlation heatmaps.
  \item PCA eigenvalue spectrum.
  \item UMAP/t-SNE for 2D/3D embeddings.
  \item Confusion matrices.
  \item Decision boundary overlays for simplified 2D spaces.
\end{itemize}
\end{frame}

% ------------------------------------------------------------
\begin{frame}{Recommended End-to-End Pipeline}
\begin{enumerate}
  \item Exploratory Data Analysis: statistics, correlations.
  \item PCA / UMAP dimensionality reduction.
  \item Train baseline classifier (LogReg, SVM).
  \item Train deep classifier (MLP, autoencoder-based).
  \item Evaluate on validation/test sets.
  \item Cluster and compare unsupervised structure to labels.
  \item Interpret latent geometry and misclassifications.
\end{enumerate}
\end{frame}

% ------------------------------------------------------------
\begin{frame}{Summary}
\begin{itemize}
  \item The paper aligns with classification + representation learning.
  \item Topic modeling is not suitable for this dataset/task.
  \item Both supervised and unsupervised ML pipelines apply.
  \item Latent space analysis (PCA/UMAP) is critically important.
  \item The proposed pipeline provides a reproducible ML framework.
\end{itemize}
\end{frame}

% ------------------------------------------------------------
\end{document}

