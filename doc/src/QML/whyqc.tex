\documentclass{beamer}
\usepackage[utf8]{inputenc}
\usepackage{amsmath, amsfonts, amssymb, bm}
\usepackage{physics}
\usepackage{graphicx}
\usepackage{hyperref}
\usepackage{tikz}
\usetheme{Frankfurt} % You can change the theme as you like
\usecolortheme{seagull}

\title[Quantum Entanglement]{\textbf{Advantages of Quantum Entanglement}}
\author{Quantum Information Lecture Series}
\institute{Department of Quantum Computing}
\date{\today}

\begin{document}

%-----------------------------------------------------------
\begin{frame}
    \titlepage
\end{frame}

%-----------------------------------------------------------
\begin{frame}{Outline}
\tableofcontents
\end{frame}

%-----------------------------------------------------------
\section{Introduction to Quantum Entanglement}
\begin{frame}{What is Quantum Entanglement?}
\textbf{Quantum Entanglement} is a quantum phenomenon where two or more particles become correlated in such a way that the state of one particle directly affects the state of the other, regardless of distance.

\vspace{10pt}
\textbf{Key Features:}
\begin{itemize}
    \item Non-local correlations
    \item No classical analog
    \item Violates Bell's inequalities
\end{itemize}

\pause
\textbf{Entangled State Example:}
\[
\ket{\Phi^+} = \frac{1}{\sqrt{2}} (\ket{00} + \ket{11})
\]

- Measurement of the first qubit immediately defines the state of the second.
\end{frame}

%-----------------------------------------------------------
\section{Mathematical Formalism}
\begin{frame}{Mathematical Representation}
In quantum mechanics, a system of two qubits is represented by a tensor product:

\[
\ket{\psi} = \ket{\psi_1} \otimes \ket{\psi_2}.
\]

A state is \textbf{entangled} if it cannot be factorized:

\[
\ket{\psi} \neq \ket{\psi_1} \otimes \ket{\psi_2}.
\]

\pause
\textbf{Bell State Example:}
\[
\ket{\Phi^-} = \frac{1}{\sqrt{2}}(\ket{01} - \ket{10})
\]

- Maximal entanglement state.  
- Violation of local realism (Bell's inequality).
\end{frame}

%-----------------------------------------------------------
\section{Advantages of Quantum Entanglement}
\begin{frame}{1. Quantum Communication}
\textbf{Quantum Teleportation:}
\begin{itemize}
    \item Entanglement enables the transmission of quantum states using classical communication.
    \item No need to send the physical quantum particle.
\end{itemize}

\pause
\textbf{Formula:}
\[
\ket{\psi} = \alpha \ket{0} + \beta \ket{1} \implies \text{(Teleportation)}
\]

\textbf{Advantage:}
- Instantaneous state transfer within quantum mechanics constraints.
- Quantum networks rely on entanglement for secure communication.
\end{frame}

%-----------------------------------------------------------
\begin{frame}{2. Quantum Cryptography}
\textbf{Quantum Key Distribution (QKD):}
\begin{itemize}
    \item Entanglement ensures secure communication.
    \item Eavesdropping disturbs quantum states, revealing interception attempts.
\end{itemize}

\textbf{BBM92 Protocol (Entangled Version of BB84):}
\[
\ket{\Phi^+} = \frac{1}{\sqrt{2}}(\ket{00} + \ket{11})
\]

- Any measurement by a third party collapses the wavefunction.  
- Ensures security based on quantum mechanics, not computational hardness.

\pause
\textbf{Advantage:}
- Unconditional security guaranteed by the laws of physics.
\end{frame}

%-----------------------------------------------------------
\begin{frame}{3. Quantum Computing}
\textbf{Speedup in Quantum Algorithms:}
\begin{itemize}
    \item Entanglement provides exponential state space.
    \item Quantum parallelism arises from entangled qubits.
\end{itemize}

\textbf{Grover's Algorithm:}
\[
\mathcal{O}(\sqrt{N}) \text{ vs. } \mathcal{O}(N)
\]

\textbf{Shor's Algorithm:}
\[
\text{Factoring in } \mathcal{O}((\log N)^3)
\]

\pause
\textbf{Advantage:}
- Solves certain problems exponentially faster than classical computers.
- Exploits entanglement for quantum parallelism. 
\end{frame}

%-----------------------------------------------------------
\begin{frame}{4. Quantum Sensing and Metrology}
\textbf{Quantum Metrology:}
\begin{itemize}
    \item Uses entangled states for ultra-precise measurements.
    \item Overcomes the classical shot-noise limit.
\end{itemize}

\textbf{Heisenberg Limit:}
\[
\Delta \theta \ge \frac{1}{N},
\]

where \( N \) is the number of entangled particles.  
\pause

\textbf{Advantage:}
- Quantum entanglement improves sensitivity beyond classical limits.
- Applications in gravitational wave detection and atomic clocks.
\end{frame}

%-----------------------------------------------------------
\section{Challenges and Limitations}
\begin{frame}{Challenges of Quantum Entanglement}
\textbf{Decoherence:}
\begin{itemize}
    \item Entangled states are fragile.
    \item Interaction with the environment collapses the wavefunction.
\end{itemize}

\textbf{Scalability:}
\begin{itemize}
    \item Difficult to entangle large numbers of qubits.
    \item Error correction requires complex protocols.
\end{itemize}

\textbf{Measurement Problem:}
\begin{itemize}
    \item Measurement destroys entanglement.
    \item Trade-off between information gain and entanglement preservation.
\end{itemize}
\end{frame}

%-----------------------------------------------------------
\section{Future Perspectives}
\begin{frame}{Future Perspectives}
\textbf{Quantum Internet:}
\begin{itemize}
    \item Entanglement as a resource for global quantum networks.
\end{itemize}

\textbf{Fault-Tolerant Quantum Computing:}
\begin{itemize}
    \item Quantum error correction leveraging entanglement.
\end{itemize}

\textbf{Advanced Quantum Sensors:}
\begin{itemize}
    \item Improved sensitivity for medical and scientific applications.
\end{itemize}

\pause
\textbf{Conclusion:}
- Quantum entanglement is a fundamental resource.  
- It enables quantum supremacy in communication, computation, and sensing.
\end{frame}

%-----------------------------------------------------------
\section{References}
\begin{frame}{References}
\begin{thebibliography}{9}
\bibitem{NielsenChuang} M. Nielsen and I. Chuang, \textit{Quantum Computation and Quantum Information}, Cambridge University Press, 2000.
\bibitem{EPR} A. Einstein, B. Podolsky, and N. Rosen, "Can Quantum-Mechanical Description of Physical Reality Be Considered Complete?", Physical Review, 1935.
\bibitem{Bell1964} J. S. Bell, "On the Einstein-Podolsky-Rosen paradox", Physics, 1964.
\end{thebibliography}
\end{frame}

\end{document}
