\documentclass[10pt]{beamer}

% Basic setup
\usetheme{default}
\usecolortheme{default}
\usefonttheme{professionalfonts}

\usepackage{amsmath}
\usepackage{tikz}
\usetikzlibrary{positioning,fit,shapes.geometric}

\title{Physical vs.\ Logical Qubits}
\author{Notes by MHJ}
\date{October 2023}

\begin{document}

% -------------------------------------------------------------
\begin{frame}
 \titlepage
\end{frame}

% -------------------------------------------------------------
\begin{frame}{What is a physical qubit?}
 \begin{itemize}
   \item A \textbf{physical qubit} is a concrete, hardware implementation of a two-level system:
     \begin{itemize}
       \item superconducting transmon, trapped ion, spin in a quantum dot, NV center, photon, \dots
     \end{itemize}
   \item Each physical qubit:
     \begin{itemize}
       \item has a finite coherence time (T$_1$, T$_2$),
       \item suffers from bit-flip and phase-flip errors,
       \item is directly controlled by microwave/laser pulses or other hardware controls.
     \end{itemize}
   \item Useful for small experiments, but error rates are typically too high for long algorithms.
 \end{itemize}
\end{frame}

% -------------------------------------------------------------
\begin{frame}{What is a logical qubit?}
 \begin{itemize}
   \item A \textbf{logical qubit} is an \emph{encoded} qubit:
     \[
       \lvert 0_L\rangle, \ \lvert 1_L\rangle
     \]
     stored across many physical qubits using a quantum error-correcting code.
   \item The code allows:
     \begin{itemize}
       \item detection and correction of physical errors,
       \item reduction of the effective error rate per logical qubit,
       \item construction of fault-tolerant logical gates.
     \end{itemize}
   \item Example: surface code can require hundreds to thousands of physical qubits per logical qubit for very low logical error rates.
 \end{itemize}
\end{frame}

% -------------------------------------------------------------
\begin{frame}{From physical to logical qubits (conceptual picture)}
 \centering
 \begin{tikzpicture}[
=latex,
     qubit/.style={circle, draw, minimum size=5mm, inner sep=0pt},
     logical/.style={rounded corners=4pt, draw, thick, inner sep=4pt},
     label/.style={font=\scriptsize}
 ]

 % --- Single physical qubit (top left) ---
 \node[qubit, fill=blue!5] (phys_single) at (0,1.8) {$q$};
 \node[label, right=4pt of phys_single.east] {Single \textbf{physical} qubit};

 % --- Many physical qubits forming one logical qubit ---
 \node[label] at (-3.2,0.8) {Physical qubit layer};

 \node[qubit] (p1) at (-3,0) {};
 \node[qubit] (p2) [right=0.55cm of p1] {};
 \node[qubit] (p3) [right=0.55cm of p2] {};
 \node[qubit] (p4) [right=0.55cm of p3] {};
 \node[qubit] (p5) [right=0.55cm of p4] {};

 \node[label, above=0pt of p1] {$q_1$};
 \node[label, above=0pt of p2] {$q_2$};
 \node[label, above=0pt of p3] {$q_3$};
 \node[label, above=0pt of p4] {$q_4$};
 \node[label, above=0pt of p5] {$q_5$};

 \node[logical, fit=(p1) (p5), label={[label]below:$\lvert 0_L\rangle$}] (logical1) {};

 \node[label, right=0.5cm of logical1.east] {One \textbf{logical qubit}};

 % --- Arrow / annotation ---
 \draw[->, thick] (phys_single.south) .. controls (0,1.1) and (-1.0,0.7) .. (p3.north)
     node[midway, right, xshift=4pt, yshift=4pt, label]
     {Encoding spreads information\\over many noisy qubits};

 \end{tikzpicture}

 \vspace{0.6em}
 \begin{itemize}
   \item Logical qubits enable \textbf{fault-tolerant quantum computing}.
   \item Many logical qubits are needed for large-scale algorithms (e.g.\ Shor, chemistry, QPE).
 \end{itemize}
\end{frame}

\end{document}
