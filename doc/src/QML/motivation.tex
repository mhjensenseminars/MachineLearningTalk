\documentclass{beamer}
\usepackage[utf8]{inputenc}
\usepackage{amsmath, amssymb, bm}
\usepackage{physics}
\usepackage{graphicx}
\usepackage{hyperref}
\usepackage{tikz}
\usetheme{Madrid} % You can change the theme as you like
\usecolortheme{seagull}

\title[Quantum Computing and ML]{\textbf{Quantum Computing and Machine Learning}}
\author{Quantum Lecture Series}
\institute{Department of Quantum Information Science}
\date{\today}

\begin{document}

%-----------------------------------------------------------
\begin{frame}
    \titlepage
\end{frame}

%-----------------------------------------------------------
\begin{frame}{Outline}
\tableofcontents
\end{frame}

%-----------------------------------------------------------
\section{Introduction to Quantum Computing}
\begin{frame}{What is Quantum Computing?}
Quantum computing leverages principles of quantum mechanics to perform computations beyond classical capabilities.

\vspace{10pt}
\textbf{Key Concepts:}
\begin{itemize}
    \item \textbf{Superposition:} Qubits can exist in a combination of states.
    \item \textbf{Entanglement:} Correlation between qubits regardless of distance.
    \item \textbf{Quantum Interference:} Probability amplitudes interfere to solve problems.
\end{itemize}

\pause
\textbf{Qubit Representation:}
\[
\ket{\psi} = \alpha \ket{0} + \beta \ket{1}, \quad |\alpha|^2 + |\beta|^2 = 1
\]
\end{frame}

%-----------------------------------------------------------
\section{Introduction to Machine Learning}
\begin{frame}{What is Machine Learning?}
Machine Learning (ML) is the study of algorithms that improve through data experience.

\textbf{Types of Machine Learning:}
\begin{itemize}
    \item \textbf{Supervised Learning:} Labeled data for classification or regression.
    \item \textbf{Unsupervised Learning:} No labels; discover hidden patterns.
    \item \textbf{Reinforcement Learning:} Learning through interaction with the environment.
\end{itemize}

\pause
\textbf{ML Workflow:}
\[
\text{Data} \rightarrow \text{Model Training} \rightarrow \text{Prediction}
\]
\end{frame}

%-----------------------------------------------------------
\section{Quantum Machine Learning (QML)}
\begin{frame}{What is Quantum Machine Learning?}
\textbf{Quantum Machine Learning (QML)} integrates quantum computing with machine learning algorithms to exploit quantum advantages.

\vspace{10pt}
\textbf{Motivation:}
\begin{itemize}
    \item High-dimensional Hilbert spaces for better feature representation.
    \item Quantum parallelism for faster computation.
    \item Quantum entanglement for richer data encoding.
\end{itemize}

\pause
\textbf{Quantum Model Example:}
\[
U(\theta)\ket{x} \implies \text{Quantum Kernel for Classification}
\]
\end{frame}

%-----------------------------------------------------------
\section{Quantum Algorithms for ML}
\begin{frame}{1. Quantum Support Vector Machines (QSVM)}
\textbf{Quantum Kernel Estimation:}
\begin{itemize}
    \item Maps classical data to a quantum Hilbert space.
    \item Quantum kernel measures similarity in high-dimensional space.
\end{itemize}

\pause
\textbf{Quantum Kernel:}
\[
K(x, x') = |\braket{\psi(x) | \psi(x')}|^2
\]

\textbf{Advantage:}
- Potentially exponential speedup over classical SVMs.
\end{frame}

%-----------------------------------------------------------
\begin{frame}{2. Quantum Neural Networks (QNNs)}
\textbf{Quantum Neural Networks} replace classical neurons with parameterized quantum circuits.

\textbf{Key Concepts:}
\begin{itemize}
    \item Quantum Gates as Activation Functions.
    \item Variational Quantum Circuits (VQCs) for optimization.
\end{itemize}

\pause
\textbf{Parameterized Quantum Circuit:}
\[
U(\theta) = \prod_i R_y(\theta_i) \cdot CNOT \cdot R_x(\theta_i)
\]

\textbf{Advantage:}
- Quantum gradients enable exploration of non-convex landscapes.
\end{frame}

%-----------------------------------------------------------
\begin{frame}{3. Quantum Boltzmann Machines (QBMs)}
\textbf{Quantum Boltzmann Machines} leverage quantum mechanics to sample from a probability distribution.

\begin{itemize}
    \item Quantum tunneling aids in escaping local minima.
    \item Quantum annealing for optimization problems.
\end{itemize}

\pause
\textbf{Quantum Hamiltonian:}
\[
H = -\sum_i b_i \sigma_i^z - \sum_{ij} w_{ij} \sigma_i^z \sigma_j^z
\]

\textbf{Advantage:}
- Efficient sampling in complex probability distributions.
\end{frame}

%-----------------------------------------------------------
\section{Quantum Speedups}
\begin{frame}{Quantum Speedups in ML}
\textbf{Why Quantum?}
\begin{itemize}
    \item \textbf{Quantum Parallelism:} Process multiple states simultaneously.
    \item \textbf{Quantum Entanglement:} Correlated states for richer information.
    \item \textbf{Quantum Interference:} Constructive and destructive interference to enhance solutions.
\end{itemize}

\pause
\textbf{Example - Grover's Algorithm:}
\[
\text{Quantum Search Complexity: } O(\sqrt{N}) \text{ vs. } O(N)
\]

\textbf{Advantage:}
- Speedups in high-dimensional optimization and linear algebra problems.
\end{frame}

%-----------------------------------------------------------
\section{Challenges in Quantum Machine Learning}
\begin{frame}{Challenges and Limitations}
\textbf{1. Quantum Hardware Limitations:}
\begin{itemize}
    \item Noisy Intermediate-Scale Quantum (NISQ) devices.
    \item Decoherence and limited qubit coherence times.
\end{itemize}

\textbf{2. Data Encoding:}
\begin{itemize}
    \item Efficient embedding of classical data into quantum states.
\end{itemize}

\textbf{3. Scalability:}
\begin{itemize}
    \item Difficult to scale circuits to large datasets.
\end{itemize}
\end{frame}

%-----------------------------------------------------------
\section{Applications of QML}
\begin{frame}{Applications of Quantum Machine Learning}
\textbf{1. Quantum Chemistry:}
\begin{itemize}
    \item Simulate molecular structures with QML.
\end{itemize}

\textbf{2. Finance:}
\begin{itemize}
    \item Quantum optimization for portfolio management.
\end{itemize}

\textbf{3. Image Recognition:}
\begin{itemize}
    \item Quantum-enhanced convolutional neural networks.
\end{itemize}
\end{frame}

%-----------------------------------------------------------
\section{Future Perspectives}
\begin{frame}{Future Perspectives in QML}
\textbf{1. Fault-Tolerant Quantum Computing:}
\begin{itemize}
    \item Overcoming noise for stable quantum circuits.
\end{itemize}

\textbf{2. Hybrid Quantum-Classical Models:}
\begin{itemize}
    \item Combining quantum circuits with classical neural networks.
\end{itemize}

\textbf{3. Quantum Internet:}
\begin{itemize}
    \item Distributed quantum machine learning over quantum networks.
\end{itemize}
\end{frame}

%-----------------------------------------------------------
\section{References}
\begin{frame}{References}
\begin{thebibliography}{9}
\bibitem{NielsenChuang} M. Nielsen and I. Chuang, \textit{Quantum Computation and Quantum Information}, Cambridge University Press, 2000.
\bibitem{Biamonte2017} J. Biamonte et al., "Quantum Machine Learning", \textit{Nature}, 2017.
\bibitem{Preskill2018} J. Preskill, "Quantum Computing in the NISQ Era", \textit{Quantum}, 2018.
\end{thebibliography}
\end{frame}

\end{document}
