\documentclass[tikz,border=5pt]{standalone}
\usepackage{amsmath}
\usetikzlibrary{arrows.meta, decorations.pathmorphing}

\begin{document}
\begin{tikzpicture}[
   thick,
   amp/.style={-Latex, very thick},
   photon/.style={decorate, decoration={snake, amplitude=1pt, segment length=6pt}},
   every node/.style={font=\small}
]

% --- Left: source and double slit geometry ---

% Source
\fill (0,0) circle (1.5pt);
\node[anchor=east] at (0,0) {Source};

% Barrier with two slits
\draw (-0.2,-1.2) -- (-0.2,1.2);
\draw (2.0,-1.2) -- (2.0,1.2);

% Slits (gaps in barrier at x=2)
\draw[line width=2pt] (2,0.25) -- (2,0.55);
\draw[line width=2pt] (2,-0.55) -- (2,-0.25);

\node[anchor=south] at (2,1.2) {Two slits};

% Screen (vertical line)
\draw (6,-1.2) -- (6,1.2);
\node[anchor=west, align=left] at (6,1.2) {Screen\\(detection probability)};

% Paths from source to screen via top slit
\draw[photon] (0,0) .. controls (1,0.4) and (1.5,0.5) .. (2,0.4)
              .. controls (3,0.5) and (4,0.8) .. (6,0.8);

% Paths from source to screen via bottom slit
\draw[photon] (0,0) .. controls (1,-0.4) and (1.5,-0.5) .. (2,-0.4)
              .. controls (3,-0.5) and (4,-0.8) .. (6,-0.8);

% Dotted guide showing many arrival points
\foreach \y in {-0.9,-0.6,-0.3,0,0.3,0.6,0.9} {
   \fill (6,\y) circle (0.5pt);
}

% Label amplitudes on each path
\node[anchor=south west] at (1.2,0.5) {$\psi_1$};
\node[anchor=north west] at (1.2,-0.5) {$\psi_2$};

% Braces / explanation text
\node[align=center, anchor=north west, font=\footnotesize] at (0.1,-1.4)
{Two coherent paths\\
$\Rightarrow$ two probability \emph{amplitudes}};

% --- Middle: addition of amplitudes ---

% Arrows to amplitude sum box
\draw[amp] (3.2,0.6) -- (3.8,0.2);
\draw[amp] (3.2,-0.6) -- (3.8,-0.2);

% Box for complex amplitude sum
\draw[rounded corners, thick, fill=white] (3.8,-0.5) rectangle (5.2,0.5);
\node[align=center] at (4.5,0)
{$\psi = \psi_1 + \psi_2$\\[4pt]
(quantum\\ superposition)};

% --- Right bottom: interference pattern plot ---

% Axes for intensity vs position
\begin{scope}[shift={(7.5,-1.2)}]
   % axes
   \draw[->] (0,0) -- (3.2,0) node[anchor=west] {position on screen};
   \draw[->] (0,0) -- (0,2.2) node[anchor=south] {$|\psi|^2$};

   % interference curve ~ cos^2
   \draw[domain=0:3,smooth,variable=\x]
       plot ({\x},{2*cos(180*\x r)^2});

   % label
   \node[anchor=west, align=left] at (0.2,1.8)
   {Bright and dark\\fringes};
\end{scope}

% Arrow from screen to intensity plot
\draw[amp] (6.2,-0.9) .. controls (6.8,-1.4) and (7.0,-1.4) .. (7.4,-1.2);

% Caption
\node[font=\footnotesize, align=center] at (4.5,-2.2)
{Interference pattern arises from \textbf{phase-coherent} addition of amplitudes.\\
Classical probabilities would \emph{add}; quantum amplitudes \emph{interfere}.};

\end{tikzpicture}
\end{document}
