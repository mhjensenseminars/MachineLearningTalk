\documentclass{beamer}
\usepackage[utf8]{inputenc}
\usepackage{amsmath, amssymb, bm}
\usepackage{physics}
\usepackage{graphicx}
\usepackage{hyperref}
\usepackage{xmpmulti}
\usepackage{tikz}
\usepackage{pgfplots}
\pgfplotsset{compat=newest}
\usepackage{siunitx}
\usepackage{longtable}
\sisetup{round-mode=places,round-precision=1}
\usepackage{braket}
\usetikzlibrary{arrows.meta, shapes.misc, positioning, backgrounds}
\usetikzlibrary{calc, decorations.markings,decorations.pathmorphing}
\usetheme{Madrid} % You can change the theme as you like
\usecolortheme{seagull}
%\renewcommandCopy{\qty\SI} 



\begin{document}

\title[Quanum ML]{\textbf{Reinforcement Learning for Quantum Sensing}}
\author{Morten Hjorth-Jensen}
\institute{Department of Physics, UiO, Norway}
\date{Quantum Machine Learning: from Fundamentals to Applications, Nordita, February 2-13, 2026}


%-----------------------------------------------------------
%\begin{frame}
%    \titlepage
%\end{frame}

%-----------------------------------------------------------


\begin{frame}[plain,fragile]
\titlepage
\end{frame}


\begin{frame}[plain,fragile]
\frametitle{Inspiration and aims}

\begin{block}{Aims}
I will try to give you an overview on how we can model systems of relevance of quantum sensing, using machine learning and many-body methods. 
\end{block}
\begin{block}{Inspiration}
\begin{itemize}
\item {\bf Reinforcement Learning for Quantum Technlogy}, by Marin Bukov abd Florian Marquardt, arXiv:2601.18953
\item {\bf Model-aware reinforcement learning for high-performance Bayesian experimental design in quantum metrology}, by Federico Belliardo, Fabio Zoratti, Florian Marquardt, and Vittorio Giovannetti, Quantum 8, 1555 (2024)
\end{itemize}
\end{block}

\end{frame}



\begin{frame}[plain,fragile]
\frametitle{Thanks to many}

\begin{alertblock}{Many good friends and colleagues}
Jane Kim (ANL), Patrick Cook (MSU), Danny Jammooa (MSU), Dean Lee (MSU), Bryce Fore (ANL), Alessandro Lovato (ANL), Stefano Gandolfi (LANL), Francesco Pederiva (UniTN), Arnau Rious (Barcelona), Giuseppe Carleo (EPFL), 
Niyaz Beysengulov (EEROQ), Johannes Pollanen (EEROQ, MSU), Zachary Stewart (MSU), Jared Weidman (MSU), Angela Wilson (MSU), Francesco Massel (USN, UiO), Gunnar Lange (UiO), Viktor Svensson (UiO), Cecilie Glittum (UiO),
Jonas Flaten (UiO), Oskar Leinonen (UiO), Øyvind Sigmundson Schøyen (UiO), Stian Dysthe Bilek (UiO), and Håkon Emil Kristiansen (UiO). Excuses to those I have omitted.
\end{alertblock}


\end{frame}



\begin{frame}[plain,fragile]
\frametitle{Di Vincenzo criteria}

\begin{alertblock}{Quantum computing requirements }
\begin{enumerate}
\item A scalable physical system with well-characterized qubit

\item The ability to initialize the state of the qubits to a simple fiducial state

\item Long relevant Quantum coherence times longer than the gate operation time

\item A universal set of quantum gates

\item A qubit-specific measurement capability
\end{enumerate}

\noindent
\end{alertblock}
\end{frame}

\frame
    {
      \frametitle{Our platform, electrons on helium}
	
      \begin{footnotesize}
     \begin{columns}
       \column{5.0cm}
\begin{alertblock}{EEROQ and MSU}
\begin{enumerate}
\item Long coherence times

\item Highly connect qubits

\item Many qubits in a small area

\item CMOS compatible

\item Fast gates
\end{enumerate}
\end{alertblock}
\column{6cm}
      \begin{center}
	\rotatebox[origin=c]{-90}{\includegraphics[width=1.3\textwidth]{qcfigures/lab.jpeg}}
      \end{center}
\end{columns}
      \end{footnotesize}
    }


\frame
    {
      \frametitle{Single electrons can make great qubits}
	
      \begin{footnotesize}
     \begin{columns}
       \column{5.0cm}
\begin{alertblock}{Electrons on helium}
       At the heart is the trapping and control
       of individual electrons floating above pools of superfluid
       helium. These electrons form the qubits of our quantum
       computer, and the purity of the superfluid helium protects the
       intrinsic quantum properties of each electron. The  ultimate
       goal is to build a large-scale quantum computer based on
       quantum magnetic (spin) state of these trapped electrons.
\end{alertblock}       
\column{5cm}
      \begin{center}
	\includegraphics[width=1.2\textwidth]{qcfigures/nordicquantumfig1.png}
      \end{center}
\end{columns}
      \end{footnotesize}
    }


\frame
    {
      \frametitle{Trapping electrons in microchannels}
	
      \begin{footnotesize}
     \begin{columns}
       \column{5.0cm}
       \begin{alertblock}{Microchannels}
         Microchannels fabricated into silicon wafers are filled with superfluid helium and energized electrodes. Together with the natural electron trapping properties of superfluid helium, these allow for the precision trapping of individual or multiple electrons. The microchannels are only a few micrometers in size, or about five times smaller than the diameter of a human hair.
\end{alertblock}                
\column{5cm}
      \begin{center}
	\includegraphics[width=1.2\textwidth]{qcfigures/nordicquantumfig2.png}
      \end{center}
\end{columns}
      \end{footnotesize}
    }

\frame
    {
      \frametitle{Control and readout}
	
      \begin{footnotesize}
     \begin{columns}
       \column{5.0cm}
\begin{alertblock}{Electron control}
       Microchannel regions can store thousands of electrons, from which one can be plucked and transported to the single electron control and readout area. In this region, microwave signals will interact with the electron to perform quantum logic gate operations, which will be readout via extremely fast electronics.
\end{alertblock}       

\column{5cm}
      \begin{center}
	\includegraphics[width=1.2\textwidth]{qcfigures/nordicquantumfig3.png}
      \end{center}
\end{columns}
      \end{footnotesize}
    }


\frame
    {
      \frametitle{Operations for quantum computing}
	
      \begin{footnotesize}
     \begin{columns}
       \column{5.0cm}
\begin{alertblock}{Electrons on helium}       
  Quantum information can be encoded in a number of ways using single electrons. Currently, we are working with the side-to-side(lateral) quantum motion of the electron in the engineered trap. This motion can either be in its lowest energy state, the ground state, or in a number of higher-energy excited states. This electron motion also provides the readout capabilities for the ultimate goal of building a large-scale quantum computer based on the electron's magnetic moment (spin).
  \end{alertblock}       
\column{5cm}
      \begin{center}
	\includegraphics[width=1.2\textwidth]{qcfigures/nordicquantumfig4.png}
      \end{center}
\end{columns}
      \end{footnotesize}
    }
    
\section{Experiment and theory}




\begin{frame}[plain,fragile]
\frametitle{Qubit platforms}

\vspace{6mm}

% inline figure
\centerline{\includegraphics[width=1.2\linewidth]{qcfigures/Elhelium2.png}}

\vspace{6mm}
\end{frame}

\frame
    {
      \frametitle{Final experimental setup}
	
      \begin{footnotesize}
     \begin{columns}
       \column{5.0cm}
\begin{enumerate}
\item Microdevice where two electrons are trapped in a double-well potential created by electrodes 1-7. The read-out is provided by two superconducting resonators dispersively coupled to  electron's in-plane motional states.

\item Coupling constants from each individual electrode beneath the helium layer.

\item The electron's energy in a  double-well electrostatic potential.
\item Screened Coulomb interaction
\end{enumerate}

\column{6cm}
      \begin{center}
	\includegraphics[width=1.25\textwidth]{qcfigures/figure1.pdf}
      \end{center}
\end{columns}
      \end{footnotesize}
    }


\begin{frame}[plain,fragile]
\frametitle{Recent work}

\begin{block}{}
\begin{itemize}
\item {\bf Coulomb interaction-driven entanglement of electrons on helium}, Niyaz R. Beysengulov, Johannes Pollanen, Øyvind S. Schøyen, Stian D. Bilek, Jonas B. Flaten, Oskar Leinonen, Håkon Emil Kristiansen, Zachary J. Stewart, Jared D. Weidman, Angela K. Wilson, Morten Hjorth-Jensen, PRX Quantum 5, 030324 (2024).
\item {\bf Design and Dynamics of High-Fidelity Two-Qubit Gates with Electrons on Helium}, Oskar Leinonen, Jonas B. Flaten, Stian D. Bilek, Øyvind S. Schøyen, Morten Hjorth-Jensen, Niyaz R. Beysengulov, Zachary J. Stewart, Jared D. Weidman, Angela K. Wilson, arXiv:2509.13946, and PRA, in press.
\item {\bf Electrons on Helium and Entangled Quantum Sensors for Particle Physics}, Niyaz R. Beysengulov, Antoine Camper, Jonas B. Flaten, Morten Hjorth-Jensen, Gunnar F. Lange, Oskar Leinonen, Jan Malamant, Francesco P. Massel, Johannes Pollanen, and Heidi Sandaker, in preparation for Communications Physics.
\end{itemize}
\end{block}

\end{frame}


\begin{frame}[plain,fragile]
\frametitle{Two-qubit gates and time evolution, SWAP gate, arXiv:2509.13946}
% inline figure
\centerline{\includegraphics[width=0.65\linewidth]{qcfigures/timeevolution.png}}
\end{frame}

\begin{frame}[plain,fragile]
\frametitle{Digression I, PINNs and quantum PINNs}

\begin{block}{Quantum Neural Networks and PINNs}
%\begin{table*}[t]
\footnotesize{
  \centering
%  \begin{threeparttable}
    \begin{tabular}{c|c|c|cc|cc}
      \toprule
      $N$ & $\omega$ & DMC & PINN+BF & \%\,err & PINN+e & \%\,err \\
      \midrule
      2 & 0.001 & --- & $0.0137948(8)$ & --- & $\mathbf{0.013778(1)}$ & --- \\
    & 0.01 & ---            & $0.69125(2)$  & --- & $\mathbf{0.69036(1)}$ & --- \\
    & 0.10 & $3.55385(5)$   & $3.5549(1)$   & $+0.0295$ & $\mathbf{3.55388(5)}$ & $+0.0008$ \\
    & 0.50 & $11.78484(6)$  & $11.7895(4)$  & $+0.0395$  & $\mathbf{11.7847(2)}$ & $-0.0000$ \\
    & 1.00 & $20.15932(8)$  & $20.1610(6)$  & $+0.0083$  & $\mathbf{20.1585(3)}$ & $-0.0041$ \\
    \midrule
    12 & 0.001 & ---          & $0.515823(3)$ & --- & $\mathbf{0.515365(4)}$ & --- \\
    & 0.01 & ---           & $2.48620(5)$  & --- & $\mathbf{2.47363(4)}$ & --- \\
    & 0.10 & $12.26984(8)$ & $12.2731(2)$  & $+0.0160$ & $\mathbf{12.2718(1)}$ & --- \\
    & 0.50 & $39.1596(1)$  & $39.1786(8)$  & $+0.0485$ & $\mathbf{39.1604(3)}$ & $+0.0018$ \\
    & 1.00 & $65.7001(1)$  & $65.717(1)$   & $+0.0257$ & $\mathbf{65.69556(5)}$ & $-0.0069$ \\
    \midrule
    20 & 0.001 & ---          & --- & --- & $\mathbf{1.293033(6)}$ & --- \\
    & 0.01 & ---           & ---  & --- & $\mathbf{6.14645(5)}$ & --- \\
    & 0.10 & $29.9779(1)$ & ---  & --- & $29.9888(2)$ & $+0.0364$ \\
    & 0.50 & $93.8752(1)$  & ---  & --- & $\mathbf{93.8789(5)}$ & $+0.0040$ \\
    & 1.00 & $155.8822(1)$  & ---   & --- & $\mathbf{155.8738(7)}$ & $-0.0024$ \\
    \bottomrule
\end{tabular}
%\end{threeparttable}
%\end{table*}}
\end{block}

\end{frame}


\begin{frame}[plain,fragile]
\frametitle{Digression II, Parametric Matrix Models and Trotterization}
      \begin{footnotesize}
     \begin{columns}
       \column{5.0cm}
       \begin{alertblock}{Electrons on helium}
         \begin{itemize}
           \item   Parametric matrix models as a way to compute the Lie-Trotter formula,
             see Nature Communications {\bf 16}, 5929 (2025) , Cook, Jammooa, MHJ, Lee and Lee
           \item Extrapolated Trotter approximation for quantum computing simulations. We plot the lowest three energies of the effective Hamiltonian for the one-dimensional Heisenberg model with DM interactions versus time step $dt$. All training (diamonds) and validation (circles) samples are located away from dt = 0.
         \end{itemize}
\end{alertblock}       
\column{5cm}
      \begin{center}
	\includegraphics[width=1.3\textwidth]{qcfigures/trotter.png}
      \end{center}
\end{columns}
      \end{footnotesize}
    }
\end{frame}


\end{document}













