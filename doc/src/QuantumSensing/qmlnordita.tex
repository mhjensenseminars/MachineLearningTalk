\documentclass{beamer}
\usepackage[utf8]{inputenc}
\usepackage{amsmath, amssymb, bm}
\usepackage{physics}
\usepackage{graphicx}
\usepackage{hyperref}
\usepackage{xmpmulti}
\usepackage{tikz}
\usepackage{pgfplots}
\pgfplotsset{compat=newest}
\usepackage{siunitx}
\usepackage{longtable}
\sisetup{round-mode=places,round-precision=1}
\usepackage{braket}
\usetikzlibrary{arrows.meta, shapes.misc, positioning, backgrounds}
\usetikzlibrary{calc, decorations.markings,decorations.pathmorphing}
\usetheme{Madrid} % You can change the theme as you like
\usecolortheme{seagull}
%\renewcommandCopy{\qty\SI} 



\begin{document}

\title[Quanum ML]{\textbf{Reinforcement Learning for Quantum Sensing}}
\author{Morten Hjorth-Jensen}
\institute{Department of Physics and Center for Computing in Science Education, University of Oslo, Norway}
\date{Quantum Machine Learning: from Fundamentals to Applications, Nordita, February 2-13, 2026}


%-----------------------------------------------------------
%\begin{frame}
%    \titlepage
%\end{frame}

%-----------------------------------------------------------


\begin{frame}[plain,fragile]
\titlepage
\end{frame}

\section{Introduction to quantum technologies}
\begin{frame}[plain,fragile]
\frametitle{Inspiration and aims}

\begin{block}{Aims}
I will try to give you an overview on how we can model systems of relevance of quantum sensing, using machine learning and many-body methods.
\end{block}
\begin{block}{Inspiration}
\begin{itemize}
\item {\bf Reinforcement Learning for Quantum Technlogy}, by Marin Bukov abd Florian Marquardt, arXiv:2601.18953
\item {\bf Model-aware reinforcement learning for high-performance Bayesian experimental design in quantum metrology}, by Federico Belliardo, Fabio Zoratti, Florian Marquardt, and Vittorio Giovannetti, Quantum 8, 1555 (2024)
\end{itemize}


\end{frame}


\frame
    {
      \frametitle{New platform: electrons on helium, \url{https://eeroq.com/}}

      \begin{footnotesize}
     \begin{columns}
       \column{5.0cm}
\begin{enumerate}
\item Long coherence times

\item Highly connected qubits

\item Many and controllable qubits in a small area

\item CMOS compatible

\item Fast gates
\end{enumerate}

\column{6cm}
      \begin{center}
        \rotatebox[origin=c]{-90}{\includegraphics[width=1.3\textwidth]{qcfigures/lab.jpeg}}
      \end{center}
\end{columns}
      \end{footnotesize}
    }





\frame
    {
      \frametitle{Single electrons can make great qubits}
	
      \begin{footnotesize}
     \begin{columns}
       \column{5.0cm}

       At the heart is the trapping and control
       of individual electrons floating above pools of superfluid
       helium. These electrons form the qubits of our quantum
       computer, and the purity of the superfluid helium protects the
       intrinsic quantum properties of each electron. The  ultimate
       goal is to build a large-scale quantum computer based on
       quantum magnetic (spin) state of these trapped electrons.
\column{5cm}
      \begin{center}
	\includegraphics[width=1.2\textwidth]{qcfigures/nordicquantumfig1.png}
      \end{center}
\end{columns}
      \end{footnotesize}
    }


\frame
    {
      \frametitle{Trapping electrons in microchannels}
	
      \begin{footnotesize}
     \begin{columns}
       \column{5.0cm}
Microchannels fabricated into silicon wafers are filled with superfluid helium and energized electrodes. Together with the natural electron trapping properties of superfluid helium, these allow for the precision trapping of individual or multiple electrons. The microchannels are only a few micrometers in size, or about five times smaller than the diameter of a human hair.
\column{5cm}
      \begin{center}
	\includegraphics[width=1.2\textwidth]{qcfigures/nordicquantumfig2.png}
      \end{center}
\end{columns}
      \end{footnotesize}
    }

\frame
    {
      \frametitle{Control and readout}
	
      \begin{footnotesize}
     \begin{columns}
       \column{5.0cm}

       Microchannel regions can store thousands of electrons, from which one can be plucked and transported to the single electron control and readout area. In this region, microwave signals will interact with the electron to perform quantum logic gate operations, which will be readout via extremely fast electronics.


\column{5cm}
      \begin{center}
	\includegraphics[width=1.2\textwidth]{qcfigures/nordicquantumfig3.png}
      \end{center}
\end{columns}
      \end{footnotesize}
    }




\end{document}



\begin{frame}[plain,fragile]
\frametitle{Di Vincenzo criteria}

\begin{alertblock}{Quantum computing requirements }
\begin{enumerate}
\item A scalable physical system with well-characterized qubit

\item The ability to initialize the state of the qubits to a simple fiducial state

\item Long relevant Quantum coherence times longer than the gate operation time

\item A \textbf{universal} set of quantum gates

\item A qubit-specific measurement capability
\end{enumerate}

\noindent
\end{alertblock}
\end{frame}

\frame
    {
      \frametitle{Important properties, electrons on helium}
	
      \begin{footnotesize}
     \begin{columns}
       \column{5.0cm}
\begin{enumerate}
\item Long coherence times

\item Highly connect qubits

\item Many qubits in a small area

\item CMOS compatible

\item Fast gates
\end{enumerate}

\column{6cm}
      \begin{center}
	\rotatebox[origin=c]{-90}{\includegraphics[width=1.3\textwidth]{qcfigures/lab.jpeg}}
      \end{center}
\end{columns}
      \end{footnotesize}
    }




\frame
    {
      \frametitle{Single electrons can make great qubits}
	
      \begin{footnotesize}
     \begin{columns}
       \column{5.0cm}

       At the heart is the trapping and control
       of individual electrons floating above pools of superfluid
       helium. These electrons form the qubits of our quantum
       computer, and the purity of the superfluid helium protects the
       intrinsic quantum properties of each electron. The  ultimate
       goal is to build a large-scale quantum computer based on
       quantum magnetic (spin) state of these trapped electrons.
\column{5cm}
      \begin{center}
	\includegraphics[width=1.2\textwidth]{qcfigures/nordicquantumfig1.png}
      \end{center}
\end{columns}
      \end{footnotesize}
    }


\frame
    {
      \frametitle{Trapping electrons in microchannels}
	
      \begin{footnotesize}
     \begin{columns}
       \column{5.0cm}
Microchannels fabricated into silicon wafers are filled with superfluid helium and energized electrodes. Together with the natural electron trapping properties of superfluid helium, these allow for the precision trapping of individual or multiple electrons. The microchannels are only a few micrometers in size, or about five times smaller than the diameter of a human hair.
\column{5cm}
      \begin{center}
	\includegraphics[width=1.2\textwidth]{qcfigures/nordicquantumfig2.png}
      \end{center}
\end{columns}
      \end{footnotesize}
    }

\frame
    {
      \frametitle{Control and readout}
	
      \begin{footnotesize}
     \begin{columns}
       \column{5.0cm}

       Microchannel regions can store thousands of electrons, from which one can be plucked and transported to the single electron control and readout area. In this region, microwave signals will interact with the electron to perform quantum logic gate operations, which will be readout via extremely fast electronics.


\column{5cm}
      \begin{center}
	\includegraphics[width=1.2\textwidth]{qcfigures/nordicquantumfig3.png}
      \end{center}
\end{columns}
      \end{footnotesize}
    }


\frame
    {
      \frametitle{Operations for quantum computing}
	
      \begin{footnotesize}
     \begin{columns}
       \column{5.0cm}
Quantum information can be encoded in a number of ways using single electrons. Currently, we are working with the side-to-side(lateral) quantum motion of the electron in the engineered trap. This motion can either be in its lowest energy state, the ground state, or in a number of higher-energy excited states. This electron motion also provides the readout capabilities for the ultimate goal of building a large-scale quantum computer based on the electron's magnetic moment (spin).       
\column{5cm}
      \begin{center}
	\includegraphics[width=1.2\textwidth]{qcfigures/nordicquantumfig4.png}
      \end{center}
\end{columns}
      \end{footnotesize}
    }
    
\section{Experiment and theory}




\begin{frame}[plain,fragile]
\frametitle{Qubit platforms}

\vspace{6mm}

% inline figure
\centerline{\includegraphics[width=1.2\linewidth]{qcfigures/Elhelium2.png}}

\vspace{6mm}
\end{frame}

\frame
    {
      \frametitle{Final experimental setup}
	
      \begin{footnotesize}
     \begin{columns}
       \column{5.0cm}
\begin{enumerate}
\item (a) Microdevice where two electrons are trapped in a double-well potential created by electrodes 1-7. The read-out is provided by two superconducting resonators dispersively coupled to  electron's in-plane motional states.

\item (b) Coupling constants from each individual electrode beneath the helium layer.

\item (c+d) The electron's energy in a  double-well electrostatic potential. 
\end{enumerate}

\column{6cm}
      \begin{center}
	\includegraphics[width=0.65\textwidth]{qcfigures/figure1.png}
      \end{center}
\end{columns}
      \end{footnotesize}
    }




\begin{frame}[plain,fragile]
\frametitle{Two-qubit gates and time evolution, SWAP gate}
% inline figure
\centerline{\includegraphics[width=0.65\linewidth]{qcfigures/timeevolution.png}}
\end{frame}














\begin{frame}[plain,fragile]
\frametitle{Observations (or conclusions if you prefer)}


\begin{block}{}
\begin{itemize}
\item How do we develop insights, competences, knowledge in AI and quantum technologies  that can advance a given field?
\begin{itemize}

  \item For example: Can we use ML to find out which correlations are relevant and thereby diminish the dimensionality problem in complex interacting  many-particle systems?

  \item Can we use AI/ML in detector analysis, accelerator design, analysis of experimental data and more?

  \item Can we use AL/ML to carry out reliable extrapolations by using current experimental knowledge and current theoretical models?
\item How do we study entanglement in various quantum platforms? Can we use AI/ML for better design?

\end{itemize}

\noindent
\item The community needs to invest in relevant educational efforts and training of scientists with knowledge in AI/ML and quantum technologies

\item Most likely tons of things I have forgotten
\end{itemize}

\noindent
\end{block}
\end{frame}

\begin{frame}{More conclusions or perspectives: Selected applications of Quantum Machine Learning}
\textbf{1. Quantum mechanical many-particle systems:}
\begin{itemize}
    \item Simulate  structures in nuclei, atoms, moleculs etc with QML.
\end{itemize}

\textbf{2. Finance:}
\begin{itemize}
    \item Quantum optimization for portfolio management.
\end{itemize}

\textbf{3. Image Recognition:}
\begin{itemize}
    \item Quantum-enhanced convolutional neural networks.
\end{itemize}
\end{frame}


\begin{frame}[plain,fragile]
\frametitle{Thank you for the attention and results from references in slides)}
\begin{enumerate}

\item Bryce Fore, Jane Kim, Morten Hjorth-Jensen, Alessandro Lovato, \textbf{Investigating the crust of neutron stars with neural-network quantum states}, Communications Physics \textbf{8}, 108  (2025) and \href{{https://www.nature.com/articles/s42005-025-02015-2}}{\nolinkurl{https://www.nature.com/articles/s42005-025-02015-2}}

\item Patrick Cook, Danny Jammooa, Morten Hjorth-Jensen, Daniel D. Lee, Dean Lee, \textbf{Parametric Matrix Models}, Nature Communications  under review and \href{{https://arxiv.org/abs/2401.11694}}{\nolinkurl{https://arxiv.org/abs/2401.11694}}

\item Niyaz R. Beysengulov, Johannes Pollanen, Øyvind S. Schøyen, Stian D. Bilek, Jonas B. Flaten, Oskar Leinonen, Håkon Emil Kristiansen, Zachary J. Stewart, Jared D. Weidman, Angela K. Wilson, Morten Hjorth-Jensen, Coulomb interaction-driven entanglement of electrons on helium, PRX Quantum 5, 030324 (2024) and \href{{https://journals.aps.org/prxquantum/abstract/10.1103/PRXQuantum.5.030324}}{\nolinkurl{https://journals.aps.org/prxquantum/abstract/10.1103/PRXQuantum.5.030324}}

\end{enumerate}
\end{frame}



\begin{frame}[plain,fragile]
\frametitle{Additional references)}
\begin{enumerate}

\item Jane Kim, Gabriel Pescia, Bryce Fore, Jannes Nys, Giuseppe Carleo, Stefano Gandolfi, Morten Hjorth-Jensen, Alessandro Lovato, \textbf{Neural-network quantum states for ultra-cold Fermi gases}, Communications Physics \textbf{7}, 148 (2024) and \href{{https://www.nature.com/articles/s42005-024-01613-w}}{\nolinkurl{https://www.nature.com/articles/s42005-024-01613-w}}

\item Bryce Fore, Jane M. Kim, Giuseppe Carleo, Morten Hjorth-Jensen, Alessandro Lovato, and Maria Piarulli, \textbf{Dilute neutron star matter from neural-network quantum states}, \href{{https://journals.aps.org/prresearch/abstract/10.1103/PhysRevResearch.5.033062}}{Physical Review  Research 5, 033062 (2023)}

\item Robert Solli, Daniel Bazin, Michelle P. Kuchera, Ryan R. Strauss, Morten Hjorth-Jensen, \emph{Unsupervised Learning for Identifying Events in Active Target Experiments}, \href{{https://www.sciencedirect.com/science/article/abs/pii/S0168900221004460}}{Nuclear Instruments and Methods in Physics Research Section A \textbf{1010}, 165461, (2020)}

\end{enumerate}
\end{frame}


\appendix{Additional material}






\begin{frame}{1. Quantum Support Vector Machines (QSVM)}
\textbf{Quantum Kernel Estimation:}
\begin{itemize}
    \item Maps classical data to a quantum Hilbert space.
    \item Quantum kernel measures similarity in high-dimensional space.
\end{itemize}

\pause
\textbf{Quantum Kernel:}
\[
K(x, x') = |\braket{\psi(x) | \psi(x')}|^2
\]

\textbf{Advantage:}
- Potentially exponential speedup over classical SVMs.
\end{frame}

\begin{frame}{2. Quantum Neural Networks (QNNs)}
\textbf{Quantum Neural Networks} replace classical neurons with parameterized quantum circuits.

\textbf{Key Concepts:}
\begin{itemize}
    \item Quantum Gates as Activation Functions.
    \item Variational Quantum Circuits (VQCs) for optimization.
\end{itemize}

\pause
\textbf{Parameterized Quantum Circuit:}
\[
U(\theta) = \prod_i R_y(\theta_i) \cdot CNOT \cdot R_x(\theta_i)
\]

\textbf{Advantage:}
- Quantum gradients enable exploration of non-convex landscapes.
\end{frame}

\begin{frame}{3. Quantum Boltzmann Machines (QBMs)}
\textbf{Quantum Boltzmann Machines} leverage quantum mechanics to sample from a probability distribution.

\begin{itemize}
    \item Quantum tunneling aids in escaping local minima.
    \item Quantum annealing for optimization problems.
\end{itemize}

\pause
\textbf{Quantum Hamiltonian:}
\[
H = -\sum_i b_i \sigma_i^z - \sum_{ij} w_{ij} \sigma_i^z \sigma_j^z
\]

\textbf{Advantage:}
- Efficient sampling in complex probability distributions.
\end{frame}


\section{Future Perspectives}
\begin{frame}{Future Perspectives in QML}
\textbf{1. Fault-Tolerant Quantum Computing:}
\begin{itemize}
    \item Overcoming noise for stable quantum circuits.
\end{itemize}

\textbf{2. Hybrid Quantum-Classical Models:}
\begin{itemize}
    \item Combining quantum circuits with classical neural networks.
\end{itemize}

\textbf{3. Quantum Internet:}
\begin{itemize}
    \item Distributed quantum machine learning over quantum networks.
\end{itemize}
\end{frame}
















\begin{frame}[plain,fragile]
\frametitle{Universal approximation theorem}

The universal approximation theorem plays a central role in deep
learning.  \href{{https://link.springer.com/article/10.1007/BF02551274}}{Cybenko (1989)} showed
the following:

\begin{block}{}
Let $\sigma$ be any continuous sigmoidal function such that
\[
\sigma(z) = \left\{\begin{array}{cc} 1 & z\rightarrow \infty\\ 0 & z \rightarrow -\infty \end{array}\right.
\]
Given a continuous and deterministic function $F(\bm{x})$ on the unit
cube in $d$-dimensions $F\in [0,1]^d$, $x\in [0,1]^d$ and a parameter
$\epsilon >0$, there is a one-layer (hidden) neural network
$f(\bm{x};\bm{\Theta})$ with $\bm{\Theta}=(\bm{W},\bm{b})$ and $\bm{W}\in
\mathbb{R}^{m\times n}$ and $\bm{b}\in \mathbb{R}^{n}$, for which
\[
\vert F(\bm{x})-f(\bm{x};\bm{\Theta})\vert < \epsilon \hspace{0.1cm} \forall \bm{x}\in[0,1]^d.
\]

\end{block}
\end{frame}

\begin{frame}[plain,fragile]
\frametitle{The approximation theorem in words}

\textbf{Any continuous function $y=F(\bm{x})$ supported on the unit cube in
$d$-dimensions can be approximated by a one-layer sigmoidal network to
arbitrary accuracy.}

\href{{https://www.sciencedirect.com/science/article/abs/pii/089360809190009T}}{Hornik (1991)} extended the theorem by letting any non-constant, bounded activation function to be included using that the expectation value
\[
\mathbb{E}[\vert F(\bm{x})\vert^2] =\int_{\bm{x}\in D} \vert F(\bm{x})\vert^2p(\bm{x})d\bm{x} < \infty.
\]
Then we have
\[
\mathbb{E}[\vert F(\bm{x})-f(\bm{x};\bm{\Theta})\vert^2] =\int_{\bm{x}\in D} \vert F(\bm{x})-f(\bm{x};\bm{\Theta})\vert^2p(\bm{x})d\bm{x} < \epsilon.
\]
\end{frame}

\begin{frame}[plain,fragile]
\frametitle{More on the general approximation theorem}

None of the proofs give any insight into the relation between the
number of of hidden layers and nodes and the approximation error
$\epsilon$, nor the magnitudes of $\bm{W}$ and $\bm{b}$.

Neural networks (NNs) have what we may call a kind of universality no matter what function we want to compute.

\begin{block}{}
It does not mean that an NN can be used to exactly compute any function. Rather, we get an approximation that is as good as we want. 
\end{block}
\end{frame}

\begin{frame}[plain,fragile]
\frametitle{Class of functions we can approximate}

\begin{block}{}
The class of functions that can be approximated are the continuous ones.
If the function $F(\bm{x})$ is discontinuous, it won't in general be possible to approximate it. However, an NN may still give an approximation even if we fail in some points.
\end{block}
\end{frame}

\end{document}

%-----------------------------------------------------------
\section{Future Perspectives}
\begin{frame}{Future Perspectives}
\textbf{Quantum Internet:}
\begin{itemize}
    \item Entanglement as a resource for global quantum networks.
\end{itemize}

\textbf{Fault-Tolerant Quantum Computing:}
\begin{itemize}
    \item Quantum error correction leveraging entanglement.
\end{itemize}

\textbf{Advanced Quantum Sensors:}
\begin{itemize}
    \item Improved sensitivity for medical and scientific applications.
\end{itemize}

\pause
\textbf{Conclusion:}
- Quantum entanglement is a fundamental resource.  
- It enables quantum supremacy in communication, computation, and sensing.
\end{frame}








