\documentclass{beamer}
\usepackage[utf8]{inputenc}
\usepackage{amsmath, amssymb, bm}
\usepackage{physics}
\usepackage{graphicx}
\usepackage{hyperref}
\usepackage{xmpmulti}
\usepackage{tikz}
\usepackage{pgfplots}
\pgfplotsset{compat=newest}
\usepackage{siunitx}
\usepackage{longtable}
\sisetup{round-mode=places,round-precision=1}
\usepackage{braket}
\usetikzlibrary{arrows.meta, shapes.misc, positioning, backgrounds}
\usetikzlibrary{calc, decorations.markings,decorations.pathmorphing}
\usetheme{Madrid} % You can change the theme as you like
\usecolortheme{seagull}
%\renewcommandCopy{\qty\SI} 



\begin{document}

\title{\textbf{Reinforcement Learning for Quantum Sensing}}
\author{Morten Hjorth-Jensen}
\institute{Department of Physics, UiO, Norway}
\date{Quantum Machine Learning: from Fundamentals to Applications, Nordita, February 2-13, 2026}


%-----------------------------------------------------------
%\begin{frame}
%    \titlepage
%\end{frame}

%-----------------------------------------------------------


\begin{frame}[plain,fragile]
\titlepage
\end{frame}


\begin{frame}[plain,fragile]
\frametitle{Aims}

\begin{alertblock}{Machine learning for Quantum}
I will try to give you an overview on how we can model systems of relevance of quantum sensing, using machine learning and many-body methods. Slides and jupyter-notebooks at \url{https://github.com/mhjensenseminars/MachineLearningTalk/tree/master/doc/src/QuantumSensing}
\end{alertblock}

\end{frame}


\begin{frame}[plain,fragile]
\frametitle{Inspiration}

\begin{alertblock}{Inspiration}
\begin{itemize}
\item {\bf Reinforcement Learning for Quantum Technlogy}, by Marin Bukov and Florian Marquardt, arXiv:2601.18953
\item {\bf Model-aware reinforcement learning for high-performance Bayesian experimental design in quantum metrology}, by Federico Belliardo, Fabio Zoratti, Florian Marquardt, and Vittorio Giovannetti, Quantum 8, 1555 (2024)
\item {\bf Hybrid quantum-classical reinforcement learning in latent observation spaces}, Daniel T. R. Nagy, Csaba Czaban, Bence Bako, Peter Haga, Zsófia Kallus, and Zoltan Zimborás  Quantum Machine Intelligence {\bf 7}, 88 (2025).
\end{itemize}
\end{alertblock}

\end{frame}



\begin{frame}[plain,fragile]
\frametitle{Thanks to many}

\begin{alertblock}{Many good friends and colleagues}
Jane Kim (ANL), Patrick Cook (MSU), Danny Jammooa (MSU), Dean Lee (MSU), Ryan LaRose (MSU), Bryce Fore (ANL), Alessandro Lovato (ANL), Stefano Gandolfi (LANL), Francesco Pederiva (UniTN), Arnau Rious (Barcelona), Giuseppe Carleo (EPFL), 
Niyaz Beysengulov (EEROQ), Johannes Pollanen (EEROQ, MSU), Zachary Stewart (MSU), Jared Weidman (MSU), Angela Wilson (MSU), Francesco Massel (USN, UiO), Gunnar Lange (UiO), Viktor Svensson (UiO), Cecilie Glittum (UiO),
Jonas Flaten (UiO), Oskar Leinonen (UiO), Øyvind Sigmundson Schøyen (UiO), Stian Dysthe Bilek (UiO), and Håkon Emil Kristiansen (UiO). Excuses to those I have omitted.
\end{alertblock}


\end{frame}



\begin{frame}[plain,fragile]
\frametitle{Di Vincenzo criteria}

\begin{alertblock}{Quantum computing requirements }
\begin{enumerate}
\item A scalable physical system with well-characterized qubit

\item The ability to initialize the state of the qubits to a simple fiducial state

\item Long relevant quantum coherence times longer than the gate operation time

\item A universal set of quantum gates

\item A qubit-specific measurement capability
\end{enumerate}
\end{alertblock}
\end{frame}

\begin{frame}
\frametitle{Our platform, electrons on helium}
  {
      
	
      \begin{footnotesize}
     \begin{columns}
       \column{5.0cm}
\begin{alertblock}{EEROQ and MSU}
\begin{enumerate}
\item Long coherence times

\item Highly connect qubits

\item Many qubits in a small area

\item CMOS compatible

\item Fast gates
\end{enumerate}
\end{alertblock}
\column{6cm}
      \begin{center}
	\rotatebox[origin=c]{-90}{\includegraphics[width=1.3\textwidth]{qcfigures/lab.jpeg}}
      \end{center}
\end{columns}
      \end{footnotesize}
    }
\end{frame}


\begin{frame}
      \frametitle{Single electrons can make great qubits}
  {
      \begin{footnotesize}
     \begin{columns}
       \column{5.0cm}
\begin{alertblock}{Electrons on helium}
       At the heart is the trapping and control
       of individual electrons floating above pools of superfluid
       helium. These electrons form the qubits of our quantum
       computer, and the purity of the superfluid helium protects the
       intrinsic quantum properties of each electron. The  ultimate
       goal is to build a large-scale quantum computer based on
       quantum magnetic (spin) state of these trapped electrons.
\end{alertblock}       
\column{5cm}
      \begin{center}
	\includegraphics[width=1.2\textwidth]{qcfigures/nordicquantumfig1.png}
      \end{center}
\end{columns}
      \end{footnotesize}
    }
\end{frame}


\begin{frame}
      \frametitle{Trapping electrons in microchannels}
  {

	
      \begin{footnotesize}
     \begin{columns}
       \column{5.0cm}
       \begin{alertblock}{Microchannels}
         Microchannels fabricated into silicon wafers are filled with superfluid helium and energized electrodes. Together with the natural electron trapping properties of superfluid helium, these allow for the precision trapping of individual or multiple electrons. The microchannels are only a few micrometers in size, or about five times smaller than the diameter of a human hair.
\end{alertblock}                
\column{5cm}
      \begin{center}
	\includegraphics[width=1.2\textwidth]{qcfigures/nordicquantumfig2.png}
      \end{center}
\end{columns}
      \end{footnotesize}
    }
\end{frame}
\begin{frame}
      \frametitle{Control and readout}
  {

	
      \begin{footnotesize}
     \begin{columns}
       \column{5.0cm}
\begin{alertblock}{Electron control}
       Microchannel regions can store thousands of electrons, from which one can be plucked and transported to the single electron control and readout area. In this region, microwave signals will interact with the electron to perform quantum logic gate operations, which will be readout via extremely fast electronics.
\end{alertblock}       

\column{5cm}
      \begin{center}
	\includegraphics[width=1.2\textwidth]{qcfigures/nordicquantumfig3.png}
      \end{center}
\end{columns}
      \end{footnotesize}
    }
\end{frame}

\begin{frame}
      \frametitle{Operations for quantum computing}
  {

	
      \begin{footnotesize}
     \begin{columns}
       \column{5.0cm}
\begin{alertblock}{Electrons on helium}       
  Quantum information can be encoded in a number of ways using single electrons. Currently, we are working with the side-to-side(lateral) quantum motion of the electron in the engineered trap. This motion can either be in its lowest energy state, the ground state, or in a number of higher-energy excited states. This electron motion also provides the readout capabilities for the ultimate goal of building a large-scale quantum computer based on the electron's magnetic moment (spin).
  \end{alertblock}       
\column{5cm}
      \begin{center}
	\includegraphics[width=1.2\textwidth]{qcfigures/nordicquantumfig4.png}
      \end{center}
\end{columns}
      \end{footnotesize}
    }
\end{frame}    


\begin{frame}
\frametitle{Qubit platforms}

\vspace{6mm}

% inline figure
\centerline{\includegraphics[width=1.2\linewidth]{qcfigures/Elhelium2.png}}

\vspace{6mm}
\end{frame}

\begin{frame}
      \frametitle{Final experimental and theoretical  setup for Coulomb entanglement}
  {

	
      \begin{footnotesize}
     \begin{columns}
       \column{5.0cm}
\begin{enumerate}
\item Microdevice where two electrons are trapped in a double-well potential created by electrodes 1-7. The read-out is provided by two superconducting resonators dispersively coupled to  electron's in-plane motional states.

\item Coupling constants from each individual electrode beneath the helium layer.

\item The electron's energy in a  double-well electrostatic potential.
\item Screened Coulomb interaction
\end{enumerate}

\column{6cm}
      \begin{center}
	\includegraphics[width=1.15\textwidth]{qcfigures/figure1.pdf}
      \end{center}
\end{columns}
      \end{footnotesize}
    }
\end{frame}

\begin{frame}[plain,fragile]
\frametitle{Recent work}

\begin{block}{}
\begin{itemize}
\item {\bf Coulomb interaction-driven entanglement of electrons on helium}, Niyaz R. Beysengulov, Johannes Pollanen, Øyvind S. Schøyen, Stian D. Bilek, Jonas B. Flaten, Oskar Leinonen, Håkon Emil Kristiansen, Zachary J. Stewart, Jared D. Weidman, Angela K. Wilson, Morten Hjorth-Jensen, PRX Quantum 5, 030324 (2024).
\item {\bf Design and Dynamics of High-Fidelity Two-Qubit Gates with Electrons on Helium}, Oskar Leinonen, Jonas B. Flaten, Stian D. Bilek, Øyvind S. Schøyen, Morten Hjorth-Jensen, Niyaz R. Beysengulov, Zachary J. Stewart, Jared D. Weidman, Angela K. Wilson, arXiv:2509.13946, and PRA, in press.
\item {\bf Electrons on Helium and Entangled Quantum Sensors for Particle Physics}, Niyaz R. Beysengulov, Antoine Camper, Jonas B. Flaten, Morten Hjorth-Jensen, Gunnar F. Lange, Oskar Leinonen, Jan Malamant, Francesco P. Massel, Johannes Pollanen, and Heidi Sandaker, in preparation for Communications Physics.
\end{itemize}
\end{block}

\end{frame}


%\begin{frame}[plain,fragile]
%\frametitle{Two-qubit gates and time evolution, SWAP gate, arXiv:2509.13946}
% inline figure
%\centerline{\includegraphics[width=0.65\linewidth]{qcfigures/timeevolution.png}}
%\end{frame}

\begin{frame}[plain,fragile]
\frametitle{Digression I, PINNs and quantum PINNs: harmonic oscillator quantum dot}

\begin{block}{Quantum Neural Networks and PINNs}
%\begin{table*}[t]
\footnotesize{
  \centering
%  \begin{threeparttable}
    \begin{tabular}{c|c|c|cc|cc}
      \toprule
      $N$ & $\omega$ & DMC & PINN+BF & \%\,err & PINN+CTNN & \%\,err \\
      \midrule
      2 & 0.001 & --- & $0.0137948(8)$ & --- & $\mathbf{0.013778(1)}$ & --- \\
    & 0.01 & ---            & $0.69125(2)$  & --- & $\mathbf{0.69036(1)}$ & --- \\
    & 0.10 & $3.55385(5)$   & $3.5549(1)$   & $+0.0295$ & $\mathbf{3.55388(5)}$ & $+0.0008$ \\
    & 0.50 & $11.78484(6)$  & $11.7895(4)$  & $+0.0395$  & $\mathbf{11.7847(2)}$ & $-0.0000$ \\
    & 1.00 & $20.15932(8)$  & $20.1610(6)$  & $+0.0083$  & $\mathbf{20.1585(3)}$ & $-0.0041$ \\
    \midrule
    12 & 0.001 & ---          & $0.515823(3)$ & --- & $\mathbf{0.515365(4)}$ & --- \\
    & 0.01 & ---           & $2.48620(5)$  & --- & $\mathbf{2.47363(4)}$ & --- \\
    & 0.10 & $12.26984(8)$ & $12.2731(2)$  & $+0.0160$ & $\mathbf{12.2718(1)}$ & --- \\
    & 0.50 & $39.1596(1)$  & $39.1786(8)$  & $+0.0485$ & $\mathbf{39.1604(3)}$ & $+0.0018$ \\
    & 1.00 & $65.7001(1)$  & $65.717(1)$   & $+0.0257$ & $\mathbf{65.69556(5)}$ & $-0.0069$ \\
    \midrule
    20 & 0.001 & ---          & --- & --- & $\mathbf{1.293033(6)}$ & --- \\
    & 0.01 & ---           & ---  & --- & $\mathbf{6.14645(5)}$ & --- \\
    & 0.10 & $29.9779(1)$ & ---  & --- & $29.9888(2)$ & $+0.0364$ \\
    & 0.50 & $93.8752(1)$  & ---  & --- & $\mathbf{93.8789(5)}$ & $+0.0040$ \\
    & 1.00 & $155.8822(1)$  & ---   & --- & $\mathbf{155.8738(7)}$ & $-0.0024$ \\
    \bottomrule
\end{tabular}
%\end{threeparttable}
%\end{table*}}
\end{block}

\end{frame}

\begin{frame}[plain,fragile]
\frametitle{What is backflow?}}
\begin{alertblock}{Learned nonlocal transformations of coordinates}
Backflow in neural quantum state calculations is a learned, nonlocal
transformation of particle coordinates or features that allows the
wavefunction to encode collective many-body correlations and accurate
fermionic nodal structures far beyond mean-field or Jastrow forms.
\end{alertblock}
\begin{alertblock}{CTNN backflow}

CTNN backflow is a symmetry-preserving, convolutional realization of
backflow in neural quantum states, where effective coordinates or
features are generated by translation-invariant neural networks to
encode collective many-body correlations and improve fermionic nodal
structure efficiently.
\end{alertblock}

\end{frame}
\begin{frame}[plain,fragile]
\frametitle{Digression II, Parametric Matrix Models and Trotterization}
      \begin{footnotesize}
     \begin{columns}
       \column{5.0cm}
       \begin{alertblock}{Parametric matrix models}
         \begin{itemize}
           \item   Parametric matrix models (PMMs) as a way to compute the Lie-Trotter formula,
             see Nature Communications {\bf 16}, 5929 (2025), Cook, Jammooa, MHJ, Lee and Lee
           \item Extrapolated Trotter approximation for quantum computing simulations. We plot the lowest three energies of the effective Hamiltonian for the one-dimensional Heisenberg model with DM interactions versus time step $dt$. All training (diamonds) and validation (circles) samples are located away from dt = 0.
         \end{itemize}
\end{alertblock}       
\column{5cm}
      \begin{center}
	\includegraphics[width=1.2\textwidth]{qcfigures/trotter.png}
      \end{center}
\end{columns}
      \end{footnotesize}
    }
\end{frame}



\begin{frame}{Two-Electron 2D Schrödinger via DVR and CI with Slater Determinants}

\textbf{Key approach for identical fermions:}
\begin{itemize}
\item We construct the two-electron basis using \textbf{Slater determinants} that include both spatial and spin degrees of freedom
\item Each basis state is a properly antisymmetrized combination: $|\psi\rangle = |\phi_a \sigma_i; \phi_b \sigma_j\rangle$ where the total wavefunction is antisymmetric under particle exchange
\item This naturally separates states into singlet (spatially symmetric) and triplet (spatially antisymmetric) configurations
\item We include the full 2D soft Coulomb interaction and diagonalize the Hamiltonian in this antisymmetrized basis
\item We compute both spin expectation values ($\langle S^2 \rangle$, $\langle S_z \rangle$) and entanglement entropy for each eigenstate
\end{itemize}

\end{frame}

\begin{frame}{Double-Well Trap Potential}

We use an arbitrary 2D potential $V(x,y)$ for the single electrons, meant to represent the electrostatic trapping potential from electrodes beneath the helium surface. For simplicity we demonstrate with an analytic double-well form:

\begin{itemize}
\item Quartic double well in x: $V_x = k(x^2 - a^2)^2$
\item Harmonic confinement in y: $V_y = \frac{1}{2}k_y y^2$
\end{itemize}
To allow specification of a 2D electrode geometry and voltages, one would solve the 3D Laplace equation for the electrode layout to find the coupling functions $\kappa_i(x,y)$ on the helium surface. Then the total trap potential is

$$V(x,y) = \sum_i \kappa_i(x,y)\,V_i$$


\end{frame}

\begin{frame}[plain,fragile]
\frametitle{Harmonic oscillator double well, no anharmonic terms}

%\vspace{6mm}

\centerline{\includegraphics[width=0.9\linewidth]{qcfigures/figure_001.png}}

%\vspace{6mm}
\end{frame}



\begin{frame}{Two-Electron Slater Determinant Basis for Identical Fermions}

For \textbf{indistinguishable fermions}, we must construct properly antisymmetrized two-electron states using Slater determinants. Each basis state includes both spatial orbitals $\phi_i(\mathbf{r})$ and spin states $\sigma \in \{\uparrow, \downarrow\}$.

A two-electron Slater determinant is:
$$|\phi_a\sigma_i, \phi_b\sigma_j\rangle = \frac{1}{\sqrt{2}}\begin{vmatrix} \phi_a(\mathbf{r}_1)\sigma_i(1) & \phi_b(\mathbf{r}_1)\sigma_j(1) \\ \phi_a(\mathbf{r}_2)\sigma_i(2) & \phi_b(\mathbf{r}_2)\sigma_j(2) \end{vmatrix}$$

For each eigenstate in our Slater determinant basis, we compute:
1. \textbf{Spin expectation values}: $\langle S^2 \rangle$ and $\langle S_z \rangle$
2. \textbf{Total entanglement entropy}: Via Schmidt decomposition including both spin and spatial degrees of freedom

\end{frame}

\begin{frame}{Why compute total (spatial + spin) entanglement?}

For \textbf{indistinguishable fermions}, the spatial and spin degrees of freedom are fundamentally coupled by antisymmetrization:
\begin{itemize}
\item \textbf{Singlet states}: Spatially symmetric × spin antisymmetric
\item \textbf{Triplet states}: Spatially antisymmetric × spin symmetric
\end{itemize}
This coupling means that \textbf{pure spatial entanglement alone would often be zero} because:
\begin{itemize}
\item For a pure singlet in orbital |a,a⟩, there's only one spatial configuration (both electrons in the same orbital)
\item The entanglement comes from the spin antisymmetry: $(|\uparrow\downarrow\rangle - |\downarrow\uparrow\rangle)/\sqrt{2}$
\end{itemize}
The \textbf{total entanglement} (including both spin and spatial) captures the full quantum correlations between the two particles and is the physically meaningful measure for identical fermions.

\end{frame}


\begin{frame}[plain,fragile]
\frametitle{States and Entanglement}

%\vspace{6mm}

\centerline{\includegraphics[width=1.0\linewidth]{qcfigures/figure_004.png}}

%\vspace{6mm}
\end{frame}


\begin{frame}[plain,fragile]{Time-Dependent Magnetic Field Perturbation and Quantum Sensing}

We now add a \textbf{time-dependent magnetic field perturbation} $B(t)$ acting on one of the electron spins through the Pauli $\sigma_z$ operator:

$$H_{\text{pert}}(t) = \frac{g\mu_B}{2} B(t) \sigma_z^{(1)}$$

where we set $g\mu_B/2 = 1$ for simplicity (can be rescaled). This perturbation:
\begin{itemize}
\item Acts only on \textbf{electron 1}
\item Causes \textbf{Zeeman splitting} between $|\uparrow\rangle$ and $|\downarrow\rangle$ states
\item Induces a \textbf{phase} $\phi(t) = \int_0^t B(t') dt'$
\end{itemize}

By acting asymmetrically on an entangled singlet, the magnetic field
converts an otherwise hidden phase into a measurable singlet–triplet
population imbalance, allowing the phase—and hence the field—to be
inferred with enhanced quantum sensitivity.

\end{frame}

\begin{frame}[plain,fragile]{Physical Picture:}

Starting with an \textbf{entangled singlet state}:
$$|S\rangle = \frac{1}{\sqrt{2}}\left(|\uparrow\downarrow\rangle - |\downarrow\uparrow\rangle\right)$$

The magnetic field on electron 1 causes:
\begin{itemize}
\item \textbf{Phase accumulation}: $|\uparrow\rangle_1 \to e^{i\phi(t)/2}|\uparrow\rangle_1$, $|\downarrow\rangle_1 \to e^{-i\phi(t)/2}|\downarrow\rangle_1$
\item  \textbf{Singlet → Triplet leakage}: The singlet mixes with the triplet $T_0 = (|\uparrow\downarrow\rangle + |\downarrow\uparrow\rangle)/\sqrt{2}$
\item \textbf{Measurable signal}: By monitoring the triplet population, we can \textbf{measure $\phi(t)$ and thus $B(t)$}
\end{itemize}
This is a \textbf{quantum sensing} protocol where entanglement enhances sensitivity to weak fields! Here we apply (for the sake og demonstration only) a weak oscillating field: $B(t) = B_0 \sin(\omega t)$.

\end{frame}

\begin{frame}[plain,fragile]{Quantum Fisher Information (QFI)}

The \textbf{Quantum Fisher Information} $F_Q(\phi)$ quantifies the sensitivity of the quantum state to the phase $\phi$:

$$F_Q(\phi) = 4\left(\langle \partial_\phi \psi | \partial_\phi \psi \rangle - |\langle \psi | \partial_\phi \psi \rangle|^2\right)$$

The \textbf{quantum Cramér-Rao bound} sets the fundamental limit on phase estimation:
$$\Delta \phi \geq \frac{1}{\sqrt{N_{\text{meas}} F_Q(\phi)}}$$

where $N_{\text{meas}}$ is the number of measurements. Higher QFI → better sensitivity!

\end{frame}

\begin{frame}{Quantum Sensing Metrics}

\begin{itemize}
    \item \textbf{Field amplitude:}
    \[
    B_0 = 0.1
    \]

    \item \textbf{Field frequency:}
    \[
    \omega = 2.0
    \]

    \item \textbf{Maximum accumulated phase:}
    \[
    \phi_{\max} = 0.0999
    \]

    \item \textbf{Singlet $\rightarrow$ Triplet leakage:}
    \[
    P_{S \to T} = 0.0099
    \]
\end{itemize}

\vspace{0.4cm}

\begin{block}{Physical Interpretation}
This singlet--triplet leakage enables indirect measurement of the
time-dependent magnetic field $B(t)$ through spin-state readout.
\end{block}

\end{frame}

\begin{frame}{Quantum Metrological Advantage I}

  \begin{block}{Enhanced sensitivity}
\begin{itemize}
    \item The entangled \textbf{singlet state} provides enhanced sensitivity
    to the time-dependent magnetic field $B(t)$ compared to separable states.
\end{itemize}
\end{block}

\begin{block}{Quantum Fisher Information}
For the optimized sensing protocol, the quantum Fisher information is
\[
F_Q = 3107.42.
\]
\end{block}

\vspace{0.4cm}
\end{frame}

\begin{frame}{Quantum Metrological Advantage II}
\begin{block}{Phase Estimation Precision}
\begin{itemize}
    \item \textbf{Shot-noise limit:}
    \[
    \Delta \phi_{\text{SQL}} \sim \frac{1}{\sqrt{N}}
    \]

    \item \textbf{Heisenberg limit (ideal entanglement):}
    \[
    \Delta \phi_{\text{HL}} \sim \frac{1}{N}
    \]

    \item \textbf{Achieved sensitivity:}
    \[
    \Delta \phi \sim \frac{1}{\sqrt{F_QN}}
    = \frac{1}{\sqrt{3107.42\,N}}
    \]
\end{itemize}
\end{block}

\end{frame}


\begin{frame}[plain,fragile]
\frametitle{Singlet-triplet leakage}

%\vspace{6mm}

\centerline{\includegraphics[width=1.0\linewidth]{qcfigures/figure_008.png}}

%\vspace{6mm}
\end{frame}


\begin{frame}[plain,fragile]{Reinforcement Learning for Optimal Quantum Control}

Finally we are implementing a \textbf{model-based reinforcement learning} approach to optimize the control sequence for quantum sensing. Since we \textbf{know the Hamiltonian}, we can simulate the quantum dynamics and use RL to find optimal control strategies.

\end{frame}

\begin{frame}[plain,fragile]{RL Framework:}

\begin{alertblock}{Sequential Decision Process}
\begin{itemize}    
\item \textbf{State}: $(t, |\psi(t)\rangle, \mu_\phi, \sigma_\phi^2)$ - time, quantum state, posterior mean & variance of phase
\item \textbf{Action}: $(B_{\text{amp}}, t_{\text{pulse}}, V_{\text{trap}})$ - magnetic field amplitude, pulse duration, trap voltage
\item  \textbf{Transition}: Evolve under $B(t)\sigma_z^{(1)} + V_{\text{trap}}(x,y)$ and kinetic plus Coulomb
\item \textbf{Measurement}: Projective measurement in singlet/triplet basis
\item \textbf{Reward}: Quantum Fisher Information of final state
\end{itemize}
\end{alertblock}
\textbf{Goal}: Maximize cumulative QFI over multiple measurement cycles to achieve optimal phase estimation.

\end{frame}

\begin{frame}[plain,fragile]{RL Agent: Cross-Entropy Method (CEM)}

\begin{alertblock}{Cross entropy}
We use the \textbf{Cross-Entropy Method}, a simple yet effective model-based RL algorithm:
\begin{itemize}
\item Sample action sequences from a Gaussian distribution
\item  Evaluate each sequence by simulation
\item Select elite sequences (top performers)
\item Update distribution to concentrate on elite actions
\item Repeat until convergence
\end{itemize}
\end{alertblock}
\end{frame}

\begin{frame}[plain,fragile]
\frametitle{Reinforcement learning}

%\vspace{6mm}

\centerline{\includegraphics[width=0.9\linewidth]{qcfigures/figure_009.png}}

%\vspace{6mm}
\end{frame}



\begin{frame}[plain,fragile]{Key Insights from RL-Optimized Control}


\begin{alertblock}{The RL agent discovers strategies that:}
\begin{itemize}   
\item \textbf{Maximize QFI}: By learning optimal pulse sequences that keep the state in regimes of high sensitivity

\item \textbf{Adaptive sensing}: Adjust control based on measurement history to refine phase estimates

\item \textbf{Balance exploration vs exploitation}: 
\begin{itemize}
\item Early cycles: stronger pulses to generate detectable leakage
\item Later cycles: fine-tuned pulses to maximize information gain
\end{itemize}
\item \textbf{Sequential optimization}: Each measurement cycle informs the next, leading to cumulative information gain

\item \textbf{Outperform heuristics}: The learned policy significantly outperforms random or fixed control sequences
\end{itemize}
\end{alertblock}
This demonstrates that \textbf{model-based RL with known quantum dynamics} can discover non-intuitive control strategies that approach fundamental quantum limits of sensing!

\end{frame}

\begin{frame}[plain,fragile]{Physical Interpretation and conclusions I}

\begin{alertblock}{What we've shown:}
\begin{itemize}
\item \textbf{Singlet-Triplet Oscillations}: The magnetic field $B(t)$ on one electron causes the initially pure singlet state to oscillate between singlet and triplet components. This "leakage" is directly measurable through spin-selective detection.
\item \textbf{Phase-Dependent Signal}: The triplet population depends on the accumulated phase $\phi(t) = \int_0^t B(t') dt'$, providing a way to measure the time-integrated field.

\item \textbf{Quantum Fisher Information}: Quantifies how sensitive our entangled state is to the phase. Higher QFI means:
\begin{itemize}
\item Better precision in estimating $\phi$ (and thus $B(t)$)
\item Approach to the Heisenberg limit of quantum metrology
\item Advantage over classical (separable state) sensors
\end{itemize}
\end{itemize}
\end{alertblock}
\end{frame}

\begin{frame}[plain,fragile]{Physical Interpretation and conclusions II}

\begin{alertblock}{What we've shown:}
\begin{itemize}
\item \textbf{Excited States vs Ground State}: We use an \textbf{excited singlet state with maximum entanglement} rather than the ground state because:
\begin{itemize}
\item Higher entanglement → stronger quantum correlations
\item More sensitive response to perturbations
\item Enhanced quantum Fisher information
\item Better phase estimation precision
\item The comparison shows QFI scales with entanglement!
\end{itemize}
\item  \textbf{Applications}:
\begin{itemize}
\item \textbf{Magnetic field sensing}: Detect weak, time-varying magnetic fields
\item \textbf{NMR/ESR spectroscopy}: Enhanced signal detection
\item \textbf{Quantum metrology}: Fundamental limits of measurement precision
\item \textbf{Quantum information}: Characterizing decoherence and noise
\end{itemize}
\end{itemize}
\end{alertblock}
\end{frame}

\begin{frame}[plain,fragile]{Physical Interpretation and conclusions III}
\begin{alertblock}{Key result}

Highly-entangled excited singlet states are superior quantum sensors
compared to weakly-entangled ground states. The singlet→triplet
leakage provides a direct readout of the magnetic field, with
sensitivity quantified by the quantum Fisher information!
\end{alertblock}
\end{frame}



\begin{frame}[plain,fragile]{Physical Insights:}

This code demonstrates fundamental quantum mechanical principles:
\begin{itemize}
\item \textbf{Fermion antisymmetry}: Total wavefunction must be antisymmetric under particle exchange
\item \textbf{Exchange interaction}: Energy splitting between singlet and triplet states arises from Coulomb exchange
\item \textbf{Quantum entanglement}: Spatial and spin entanglement in two-electron systems
\end{itemize}

\end{frame}



\end{document}















